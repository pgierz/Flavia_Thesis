\chapter[Publication 2]{Light regime affects the seasonal cycle of Antarctic krill (\textit{Euphausia superba}): impacts on growth, feeding, lipid metabolism, and maturity}

Flavia Höring1,2, Mathias Teschke1, Lavinia Suberg1, So Kawaguchi3,4, and Bettina Meyer1,2,5

1 Alfred Wegener Institute Helmholtz Centre for Polar and Marine Research, Section Polar Biological Oceanography, Am Handelshafen 12, 27570 Bremerhaven, Germany

2 Carl von Ossietzky University of Oldenburg, Institute for Chemistry and Biology of the Marine Environment (ICBM), Carl-von-Ossietzky-Str. 9-11, 26111 Oldenburg, Germany

3 Australian Antarctic Division, Department of the Environment and Energy, Channel Highway, Kingston, Tasmania 7050, Australia

4 Antarctic Climate and Ecosystems Cooperative Research Centre, University of Tasmania, Private Bag 80, Hobart, Tasmania 7001, Australia

5 Helmholtz Institute for Functional Marine Biodiversity (HIFMB) at the University of Oldenburg, 26111 Oldenburg, Germany 

Can. J. Zool. 96: 1203–1213 (2018) dx.doi.org/10.1139/cjz-2017-0353 

\section{Abstract}
Light regime is an important zeitgeber for Antarctic krill (\textit{Euphausia
superba} Dana, 1850), which seems to entrain an endogenous timing system that
synchronizes its life cycle to the extreme light conditions in the Southern
Ocean. To understand the flexibility of Antarctic krill’s seasonal cycle, we
investigated its physiological and behavioural responses to different light
regimes and if an endogenous timing system was involved in the regulation of
these seasonal processes. We analysed growth, feeding, lipid content, and
maturity in a 2-year laboratory experiment simulating the latitudinal light
regimes at 52$^{\circ}$S and 66$^{\circ}$S and constant darkness under constant
food level. Our results showed that light regime affected seasonal cycles of
growth, feeding, lipid metabolism, and maturity in Antarctic krill. Seasonal
patterns of growth, feeding, and maturity persisted under constant darkness,
indicating the presence of an endogenous timing system. The maturity cycle
showed differences in critical photoperiods according to the simulated
latitudinal light regime. This suggests a flexible endogenous timing mechanism
in Antarctic krill, which may determine its response to future environmental
changes.

\section{Introduction}
Concerns are growing about the impact of global warming on the Antarctic marine
ecosystem. The observed changes in sea-ice extent and zooplankton distribution
may lead to trophic mismatches and thereby profound changes in the Southern
Ocean food web (Atkinson et al. 2004; Steinberg et al. 2015). To be able to
predict future changes, we need to better understand the adaptive potential of
polar key organisms such as the Antarctic krill (Euphausia superba Dana, 1850)
(Meyer 2010).

Antarctic krill’s success in the Southern Ocean likely originates from its
ability to synchronize its life cycle to local photoperiod and food supply. It
has evolved seasonal patterns of growth, lipid turnover, metabolic activity
(Meyer et al. 2010), and maturation (Kawaguchi et al. 2007) that bring an
evolutionary advantage to survive in an environment with strong seasonal
fluctuations of sea-ice extent, photoperiod, and primary production. These
seasonal patterns seem to vary according to latitudinal region, as it has been
observed that Antarctic krill near South Georgia (\SI{54}{\degree}S) had lower
lipid stores and higher feeding activities in winter com- pared with regions at
higher latitudes where near-constant dark- ness during winter limits food
supply (Schmidt et al. 2014). However, the mechanisms shaping these seasonal
rhythms remain poorly understood.

Photoperiod seems to play a major role in the modulation of the seasonal
rhythms of Antarctic krill. Laboratory experiments revealed that photoperiod
affected seasonal patterns of growth (Brown et al. 2010), maturity (Hirano et
al. 2003; Teschke et al. 2008; Brown et al. 2011), feeding, and metabolic
activity (Teschke et al. 2007). It is not yet clear if light regime also
promotes acclimatization to the varying seasonal conditions in different
latitudinal habitats of Antarctic krill.

An endogenous timing system may be involved in the regulation of seasonal
rhythms in Antarctic krill. Seasonal patterns of maturity were observed to
persist under constant darkness (Brown et al. 2011), indicating an endogenous
timing system that maintained the rhythm even if the zeitgeber (environmental
cue) was absent (= concept of a biological clock). Recent studies suggest that
Antarctic krill possesses a circadian clock that regulates its daily metabolic
output rhythms and is entrained by photoperiod (Mazzotta et al. 2010; Teschke
et al. 2011). However, it is unknown if the circadian clock is also involved in
the timing of seasonal events in Antarctic krill.

This study aims to investigate the effect of different light regimes on growth,
feeding, lipid metabolism, and maturity in Antarctic krill, as well as the
involvement of an endogenous timing system in the modulation of seasonal
rhythms. We analyse a unique data set from multiyear laboratory experiments
simulating different latitudinal light regimes (\SI{52}{\degree}S,
\SI{66}{\degree}S, constant darkness) and constant food supply over 2 years. We
will test (i) if light regime stimulates seasonal patterns of growth, feeding,
lipid metabolism, and maturity; (ii) if different latitudinal light regimes
cause different seasonal patterns; and (iii) if seasonal patterns persist under
constant darkness indicating an endogenous timing system.

\section{Materials and Methods}

\subsection{Antarctic krill collection and maintenance prior to the experiments}

Antarctic krill were caught with a rectangular mid-water trawl (RMT 8) on 12
February, 2013 (\SI{66}{\degree}\SI{47}{\arcmin}S,
\SI{65}{\degree}\SI{08}{\arcmin}E) during the voyage V3 12/13 of RSV Aurora
australis and on 15 January, 2015 (\SI{65}{\degree}\SI{31}{\arcmin}S,
\SI{141}{\degree}\SI{23}{\degree}E) during voyage V2 14/15. The sampling
methods are described in detail by King et al. (2003). The sampled Antarctic
krill arrived at the Australian Antarctic Division aquarium in Hobart on 22
February, 2013 and on 25 January, 2015, respectively. For acclimation and for
keeping of Antarctic krill until the start of the experiments, they were
transferred to \SI{800}{\liter} tanks (temperature \SI{0.5}{\celsius}) that
simulated the natural light regime at \SI{66}{\degree}S. A detailed description
of the Antarctic krill aquarium facility and the simulated light regime can be
found in Kawaguchi et al. (2010). 

\subsection{Photoperiodic-controlled laboratory experiments} 

Long-term laboratory experiments were conducted over a period of 2 years
starting in January 2015. Three different light regimes were tested, simulating
(1) natural light conditions at \SI{52}{\degree}S, (2) natural light conditions
at \SI{66}{\degree}S, and (3) constant darkness (DD) (Figs. 1a, 1b). For each
treatment, $250$ Antarctic krill were transferred from the \SI{800}{\liter}
acclimation tanks to a \SI{250}{\litre} experimental tank connected to a
recirculating chilled seawater system with a constant water temperature of
\SI{0.5}{\celsius}. For the initial experimental set-up, Antarctic krill
collected in 2013 were used (tanks A, B, E, F). 

However, due to increased mortality in tank A (treatment DD), an additional
tank for treatment DD (tank K) was set up in the beginning of March 2015 using
freshly caught Antarctic krill collected in 2015. The three different light
conditions were simulated within black lightproof plastic containers, one for
each experimental tank, using twin fluorescent tubes (Osram L18W/640 Cool
White) with a marine blue gel filter (Marine Blue 131; ARRI Australia Pty.
Ltd.). Light adjustment under treatments \SI{52}{\degree}S and
\SI{66}{\degree}S was carried out using a PC-controlled timer and dimming
system (winDIM version 4.0e; EEE, Portugal) with a maximum light intensity of
100 lx (photon flux = \SI{1.3}{\micro\mole\per\meter\square\per\second}) during
midday in January (corresponds to 1\% light penetration at \SI{30}{\meter}
depth).  According to the light regime, photoperiod and light-intensity
profiles were adjusted at the beginning of each month for each treatment. The
simulated light-intensity profiles for each treatment and month can be found in
Supplementary Table S1.1 

The food level was held constant to remove that effect from our experiments
because we solely wanted to identify the effect that light regime had on the
seasonal cycle of Antarctic krill. Antarctic krill were fed daily between the
hours of 0830 and 0930 and the water flow in the tanks was turned off for
approximately 2 h to ensure feeding. The food comprised three live
laboratory-cultured algae (final concentrations were
\SI{1.5e4}{\cells\per\milli\liter} of \textit{Phaeodactylum tricornutum} Bohlin, 1897,
\SI{2e4}{\cells\per\milli\liter} of \textit{Geminigera cryophila} (D.L. Taylor and C.C.
Lee) D.R.A. Hill, 1991, \SI{2.2e4}{\cells\per\milli\liter} of \textit{Pyramimonas
gelidicola} McFadden, Moestrup and Wetherbee, 1982), three types of commercial
algal paste (\SI{1e4}{\cells\per\milli\liter} of \textit{Thalassiosira
weissflogii} (Grunow) G. Fryxell and Hasle, 1977 “TW 1200TM”,
\SI{5.1e4}{\cells\per\milli\liter} of Isochrysis Parke, 1949 “Iso 1800TM”,
\SI{4.8e4}{\cells\per\milli\liter} of Pavlova Butcher, 1952 “Pavlova 1800TM”;
Reed Mariculture, USA), and two types of prawn hatchery feeds (\SI{0.5}{\gram}
of FRiPPAK FRESH \#1CAR, \SI{0.5}{\gram} of FRiPPAK FRESH \#2CD; INVE, Thailand).
Antarctic krill under treatment DD were fed in dim red light. Moults and dead
Antarctic krill were removed regularly from the tanks. 

Antarctic krill sampling of 6–10 individuals per tank and month was carried out
in the middle of each month during midday starting in February 2015 (for
treatment DD in dim red light). Due to different rates of mortality in the
tanks, the sampling scheme had to be adjusted during the course of the
experiment (Table 1) to assure sampling over the whole experimental period. Due
to the problem with increased mortality under treatment DD mentioned above, we
decided to sample tanks A and K sequentially to ensure the completion of the
experiment over the 2-year period. 

Live Antarctic krill was inspected under a stereomicroscope and the sex was
determined. Pictures of the carapace and the sexual organs (female thelycum and
male petasma) were taken with a Leica DFC 400 camera system (Leica
Microsystems, Germany). Car- apace length (tip of the rostrum to posterior
notch) and digestive gland length (longest axis through carapace) were
determined from the pictures within the Leica \code{DFC Camera} software
\code{version 7.7.1} (Leica Microsystems, Switzerland). 

After visual inspection, the sampled Antarctic krill was immediately frozen in
liquid nitrogen. Frozen samples were stored at \SI{-80}{\celsius}.

The first inspection of the sex ratio within the experimental tanks revealed
that females dominated, with proportions of 71\%– 85\% per tank. 

\subsection{Growth analysis} 

Carapace length was used as a proxy for growth in the experiments. Antarctic
krill were sampled randomly from each experimental tank; thus, a general trend
observed in the carapace length data are assumed to display the general trend
of growth. 

The data analysis was performed in \code{RStudio version 1.0.136} (RStudio Team,
2016). Before the modelling process, a Pearson’s product moment correlation was
conducted to determine a potential difference in growth pattern between male
and female Antarctic krill; thus, the need for separate models for each sex.
Due to the strong correlation (r = 0.82, p < 0.001) between males and females,
based on the mean carapace length for each sex across all treatments, data from
both sexes were combined (n = 617). To investigate the long-term trend
(variable “time”) and the seasonal variability (variable “month”) of Antarctic
krill growth for each “treatment” (light regime), a generalized additive mixed
model (GAMM) with a Gaussian distribution was used. An additive model was
chosen over a linear one to resolve the nonlinear relationship of the response
and explanatory variables. The GAMM takes the structure as specified by Hastie
and Tibshirani (1987) and was fitted using the gamm function in the mgcv
package (Wood 2006). Random effects for “tank” were included in the model to
account for potential dependencies between individuals from the same tank.
Prior to the modelling process, temporal autocorrelation was examined using the
acf function in \code{R}. Time series are often subject to latitudinal
dependencies between data points and not accounting for the autocorrelation can
result in biased estimates of model parameters (Panigada et al. 2008). As
autocorrelation was neither detected, nor evident in residual analysis during
model validation, no temporal autocorrelation term was included in the final
model. 

Smoothed terms were fitted as regression splines (variable “time”), apart for
the variable “month”, which was modelled using cyclic cubic regression splines,
setting knots manually between 1 (January) and 12 (December) to account for the
circular nature of this term. Differences in temporal pattern between the three
light regimes (\SI{52}{\degree}S, \SI{66}{\degree}S, DD) were implemented using the by- argument of the
gamm function, which allows for the creation of separate smoothers for each
level of the treatment factor (light regime) over the temporal variables
“month” and “time”. Hence, separate parameter estimates for the temporal
variables are obtained for each treatment level. To avoid overfitting, the
smooth function of the variable “month” was manually restricted to k = 5. Model
selection was conducted using manual stepwise-backward selection based on
Akaike’s information criterion (AIC) (Akaike 1981). If the addition of a term
led to an AIC decrease of >2 per degree of freedom, or an increase of the
adjusted R2, or if the term was significant, then the term was included in the
model. Model fit was examined by residual analysis.

\subsection{Feeding Analysis}

The feeding index (\%) was calculated as digestive gland length $\times$
(carapace length)-1 $\times$ 100. Data of males and females were combined
because of the strong correlation of monthly mean values (Pearson’s product
moment correlation, r = 0.95, p < 0.001). To investigate a temporal pattern in
the feeding index of Antarctic krill for each treatment, a GAMM was employed as
described above (section Growth analysis). The smooth function of the variable
“time” was manually restricted to k = 6.

\subsection{Lipid content analysis}

Every 3 months from April 2015 to July 2016, six replicate samples from each
treatment were tested for their lipid content. Lipids were extracted from the
carapace, which was separated from the frozen samples with a scalpel on dry ice
prior to extraction. Lipid extraction was performed with
dichloromethane:methanol (2:1, v:v) according to the method described by Hagen
(2000). Lipid content was determined gravimetrically and was calculated in
percentage of dry mass. One data point (sample code “Jan16\_E04”) was removed
due to the negative value of lipid content that indicated incorrect measurement
for that individual. 

Lipid content differed between male and female Antarctic krill (Pearson’s
product moment correlation of pooled monthly mean values, r = 0.26, p = 0.62);
therefore, statistical analysis was performed separately for each sex. Data for
males were not sufficient for robust modelling and only females were considered
for this analysis (n = 83). Only one tank for each time point and treatment was
available, therefore a mixed model to resolve a potential tank effect could not
be employed. For treatment DD, five samples were available from a second tank,
but these were not sufficient for the inclusion of a random effect. Therefore,
a generalized additive model (GAM) was employed to examine the temporal pattern
of female Antarctic krill lipid content, following the protocol described in
section Growth analysis. The smooth function of the variable “time” was
manually restricted to k = 6. Because the variable “month” was not significant,
it was excluded from the final model.

\subsection{Maturity Analysis}

The maturity stage of the sampled Antarctic krill was assessed by analysing
pictures of the external sexual organs according to Makarov and Denys (1980)
and Thomas and Ikeda (1987). A maturity score was assigned using the method of
Brown et al. (2010, 2011). Due to the ordinal characteristic of the maturity
scores, Pearson correlation of monthly mean values could not be per- formed
with the data set. Therefore, we visually inspected the relationship between
maturity score and hours of light in males and females. Seasonal maturity
scores differed between male and female Antarctic krill (Fig. 2); therefore,
statistical analysis was performed on females only (n = 493), as there were not
sufficient data to allow for modelling males separately. To investigate the
temporal pattern of maturity of female Antarctic krill for each treatment, a
GAMM was employed as described in section Growth analysis. Because model
residuals were autocorrelated, an auto- regressive correlation structure of the
order 1 was added, which improved model fit and resolved the dependencies
between residuals. Maturity scores are represented as whole numbers and take
values between 3 and 5. Therefore, the GAMM was initially modelled using a
Poisson distribution with a logarithmic link function between predictor and
response. Due to overdispersion, a negative binomial GAMM had to be used. The
smooth function of the variable “time” was manually restricted to k = 6. 

To examine differences in the critical photoperiod between latitudinal light
regimes 52$^{\circ}$S and 66$^{\circ}$S, a logistic regression was used. As
only full maturity was investigated, maturity scores <5 were set to zero and
full maturity (score = 5) was set to one in all samples, resulting in a data
set of zeros and ones. The relationship between full maturity of female
Antarctic krill and photoperiod was modelled with a binomial generalized linear
mixed model (GLMM) with a logit function between predictor and response and an
interaction term for factor “treatment” and continuous variable “hours of
light”. The model was fitted using the glmer function from the lme4 library. To
account for dependencies between individuals from the same tank, random effects
for “tank” were included in the model. Model fit was assessed by constructing a
receiver operating characteristic (ROC) curve using the pROC package in R,
where the area under the curve (AUC) indicates the goodness of fit (Boyce et
al. 2002). Values below 0.7 are considered poor and 1.0 represents a perfect
fit (Cumming 2000). The critical photoperiod (= photoperiod, when the
probability to be fully mature is 50\%) was predicted from the 95\% confidence
intervals.

\subsection{Data archiving} 

Processed data have been uploaded to the database PANGAEA and can be accessed
under \url{https://doi.pangaea.de/10.1594/PANGAEA.885889.}

\section{Results}

\subsection{Growth Analysis} 

Carapace length ranged from 8.1 to 19.02 mm with a mean ($\pm$SD) of 11.71 mm
($\pm$1.61 mm) across the whole data set. The GAMM (model M1; Table 2) revealed
significant seasonal and interannual patterns in growth, which were similar
across all treatments (Figs. 3a, 3b). Shrinkage was observed in the beginning
of the experiments. A significant seasonal variability with shrinkage to- wards
austral winter (June to August) and growth towards austral summer (December to
February) was observed under treatments 52$^{\circ}$S and DD (not significant
under treatment 66$^{\circ}$S).

\subsection{Feeding}

The feeding index data ranged from 25.15\% to 66.09\% with a mean ($\pm$SD) of
42.00\% ($\pm$6.58\%). 

The GAMM revealed significant changes in the feeding index over time (model M2;
Table 2). We observed an increase of the feeding index throughout the
experimental period in all treat- ments and a final stagnation in treatments
52$^{\circ}$S and DD (Figs. 4a, 4b). The seasonal trend differed between
treatments. In treatment 52$^{\circ}$S, the feeding index strongly increased
during the autumn period (March to May) with a subsequent decrease and
stabilization during the rest of the year. The seasonal trend in treatment
66$^{\circ}$S was very weak and will therefore not be described further. In
treatment DD, the feeding index increased over a longer period (March to July)
and decreased during the rest of the year.

\subsection{Lipids}

The lipid content data of males and females ranged from 2.53\% to 57.75\% with
a mean ($\pm$SD) of 17.04\% ($\pm$9.12\%). The GAM considering female lipid
content data only (model M3; Table 2) revealed significant differences in
temporal variability of lipid content be- tween the experimental treatments
(Fig. 5). Even though the variable “month” was not significant, a resembling
seasonal pattern was observed in the interannual trend under treatment
66$^{\circ}$S with an increase towards austral winter and a decrease towards
austral summer. The increase of lipid content during the second winter was much
stronger than the first winter. No significant patterns were found for
treatments 52$^{\circ}$S and DD.

\subsection{Maturity}

Implementing the negative binomial GAMM for female maturity (model M4; Table
2), we found a significant seasonal cycle of maturity under treatments
52$^{\circ}$S, 66$^{\circ}$S, and DD with sexual regression towards austral
winter and sexual re-maturation towards austral spring and summer (Figs. 6a,
6b). Significant interannual patterns differed between treatments. In
treatments 52$^{\circ}$S and 66$^{\circ}$S, a slight decrease of maturity over
the whole study period was observed. The interannual pattern in treatment DD
showed that sexual regression was only completed during the first winter of the
experiments. 

The binomial GLMM (model M5; Table 2) suggests that the variable “hours of
light” significantly affects female maturity in treatments 52$^{\circ}$S and
66$^{\circ}$S. The interaction term between “hours of light” and “treatment”
was marginally not significant. When investigating the critical photoperiod at
the probability of 50\%, differences between the treatments were found (Fig.
7). For treatment 52$^{\circ}$S, the critical photoperiod was estimated as 12.5
h of light with 95\% confidence intervals (11.86, 13.22). For treatment
66$^{\circ}$S, an estimate of 14.76 h of light with 95\% confidence intervals
(13.3, 16.3) was found.

\section{Discussion}

We present findings from the first 2-year laboratory experiments investigating
the effect of light regime and the biological clock on the seasonal cycle of
Antarctic krill. 

The observed seasonal cycles of growth, feeding, lipid metabolism, and maturity
under the simulated latitudinal light regimes suggest that light regime is an
essential zeitgeber for Antarctic krill. The occurrence of a pronounced lipid
cycle under treatment 66$^{\circ}$S and the observed differences in critical
photoperiods for the maturation cycle indicate that Antarctic krill may respond
flexibly to different latitudinal light regimes. This may represent an adaptive
mechanism to the extreme light regimes in the Southern Ocean and ensure
survival of Antarctic krill in different latitudinal habitats, especially
during winter. Moreover, seasonal patterns of growth, feeding, and maturity
persisted under constant darkness indicating the presence of an endogenous
timing system modulating these rhythms. High food supply does not suppress
endogenously driven seasonal rhythms of growth, feeding, lipid metabolism, and
maturity. 

The following considerations should be taken into account when interpreting the
findings of this study. Due to limits in space and costs for the long-term
laboratory experiments and variable mortality rates in the tanks, we had to
adjust the experimental set-up and sampling scheme accordingly. This led to a
sampling design with replication in experimental units over the full study
period for treatment 52$^{\circ}$S only. Carapace length, digestive gland
length, and maturity data from treatment 66$^{\circ}$S and partly treatment DD,
as well as the lipid content data set, may be regarded as pseudoreplicated
(Colegrave and Ruxton 2018) because the replication in experimental units over
the full study period is incomplete. We have included the random effect
“experimental tank” in our models, where appropriate, during statistical
analysis of the data to account for a potential tank effect as far as possible.
How- ever, we cannot exclude that differences in tank and replicate number may
have influenced the results of our tests. 

To interpret the response of Antarctic krill to constant darkness over the full
2-year period, we combined data from two different cohorts of Antarctic krill.
The “new” cohort was acclimated to the laboratory conditions for 1 year, before
sampling started. Preliminary analysis revealed similar trends in both cohorts
under constant darkness, which supports our assumption that both cohorts
responded similarly to the treatment. 

Moreover, we decided to solely analyse a reduced data set for lipid content
because frozen Antarctic krill samples from the 2-year experiments are very
valuable and can be used for multiple analyses. The reduced data set is
adequate to display the pronounced seasonal lipid cycle under the high
latitudinal light regime, but it may be insufficient to test for weaker
patterns in the other treatments. Since potential differences in the male
pattern were indicated and the number of males was too low to conduct a
separate analysis, we decided to analyse females only for lipid content and
maturity. 

Moreover, we presume that the observations made in the first few months of the
experiment represent a general period of acclimation to the experimental
conditions. It may explain the strong shrinkage, suppressed lipid accumulation,
and a general similarity of the data under all treatments in the beginning of
the experiments. 

Our observation of a seasonal cycle of growth confirms findings by Brown et al.
(2010) that suggest growth is influenced by light regime, independently of food
supply. For the first time, we show that Antarctic krill’s growth cycle is
endogenous and persists under constant darkness. The observed shrinkage in
autumn and winter in this study may be partly related to the maturity cycle.
Females have been observed to shrink during sexual regression (Thomas and Ikeda
1987) and Tarling et al. (2016) suggested that it might be explained by
morphometric changes due to the contraction of the ovaries. On the other hand,
the shrinkage may reflect an overwintering mechanism (Quetin and Ross 1991).
This is sup- ported by our observation of significant seasonal shrinkage under
constant darkness where we did not find a pronounced maturity cycle over the
2-year period. 

The seasonal increase of feeding in autumn, which was observed under treatment
52$^{\circ}$S, may represent an inherent strategy to be able to accumulate
enough lipid stores for winter (Hagen et al. 2001; Meyer et al. 2010). These
results partly agree with the short-term study by Teschke et al. (2007) who
observed higher clearance rates under autumn and summer light conditions
compared with constant darkness, suggesting enhanced feeding activity under
light conditions of prolonged day length. The comparability of both studies may
be limited because we solely used a morphometric index as a measure of feeding
activity. The feeding index may be biased by the strong shrinkage that occurred
in the beginning of our experiments, which could have masked a suppressed
feeding activity in the first months. In our long-term study, the seasonal
feeding trend under treatment DD resembled the other treatments with a shift of
peak feeding activity towards winter that may indicate an endogenous control of
seasonal feeding activity in Antarctic krill. The general increase of feeding
index during the experiments suggests that Antarctic krill is able to make use
of food supply throughout the whole experimental period. This observation may
also indicate a flexible feeding behaviour of Antarctic krill (Atkinson et al.
2002) that has also been observed in the field in winter (Quetin and Ross 1991;
Huntley et al. 1994; Schmidt et al. 2014). 

In our study, we observed a seasonal pattern of lipid content under treatment
66$^{\circ}$S that may be stimulated by the high latitudinal light regime. It
resembles the lipid cycle observed in the field with highest values of lipid
content in autumn and lowest values in early spring (Hagen et al. 2001; Meyer
et al. 2010). This is the first study that shows the possible influence of
light regime on the lipid cycle in Antarctic krill. The accumulation of lipid
re- serves may be adjusted according to the latitudinal light regime, which may
explain the differences observed in the field with higher lipid stores found in
regions at higher latitudes (Schmidt et al. 2014). We also observed a match of
the period of lipid depletion and re-maturation, which supports the assumption
that lipid stores may be used for the maturation process (Teschke et al. 2008). 

The effect of light regime on the maturity cycle (Hirano et al. 2003; Teschke
et al. 2008; Brown et al. 2011) is confirmed by our study. The endogenous cycle
of maturity under constant darkness has been observed in short-term experiments
before (Thomas and Ikeda 1987; Kawaguchi et al. 2007; Brown et al. 2011). We
show that this pattern does not persist during the second year under constant
darkness and suggest that the zeitgeber photoperiod is required for the
entrainment of the maturity cycle over longer periods. Results from former
experiments (Hirano et al. 2003; Brown et al. 2011) indicate that Antarctic
krill’s maturity cycle may be entrained by the timing of two contrasting
photoperiods (peak and trough light regimes). 

To study potential differences in the physiological response of Antarctic krill
to different latitudinal light regimes, we used the critical photoperiod
(defines the day length when 50\% of the population shift from one state to
another, here maturity) as an indicator to determine the time of the year that
is a turning point in the seasonal cycle. However, using critical photoperiod,
we cannot give rise to any conclusion regarding the mechanism of entrainment of
these rhythms. We observed that the critical photo- period for maturity
differed between latitudinal light regimes, being higher under the high
latitudinal light regime. An increase of critical photoperiod with latitude has
also been found in insects in relation to diapause (Bradshaw and Holzapfel
2007; Tyukmaeva et al. 2011; Hut et al. 2013). Organisms with higher critical
photo- periods have an adaptive advantage under the extreme seasonal changes of
photoperiod at higher latitudes where they have to prepare early enough to
ensure survival during winter. Specifically, a higher critical photoperiod for
maturity implies that Ant- arctic krill is able to undertake the critical stage
of sexual regression and re-maturation during the time of the year when
photoperiods are longer compared with regions at lower latitudes. In regions
with extreme changes of photoperiod and severe winter conditions, this adaptive
mechanism may ensure that Ant- arctic krill prepares early enough for winter
and keeps up energy- saving mechanisms long enough. 

Antarctic krill’s flexibility in adjusting its photoperiodic response to a wide
range of latitudinal light regimes may be advantageous under future climate
change, as a southward migration trend of Antarctic krill to higher latitudes
at the western Antarctic Peninsula has been reported (Ross et al. 2014). Still,
changes in sea-ice dynamics, such as the timing of sea-ice formation or melt,
may lead to mismatches in the timing of critical life-cycle events (Clarke et
al. 2007). For instance, an earlier phytoplankton bloom associated with earlier
sea-ice melt may influence the survival and reproductive success of Antarctic
krill. Therefore, its potential to adapt to future environmental changes may
also depend on its genetic flexibility in adjusting its photoperiodic response
and the timing of critical life-cycle events (Bradshaw and Holzapfel 2007). 

Our findings support the assumption of a circannual timing system synchronized
by light regime in Antarctic krill (Meyer 2011). The modulation of seasonal
rhythms of growth, feeding, lipid metabolism, and maturity happen independently
of constant food supply, indicating an inherent mechanism in Antarctic krill
that regulates the timing of these processes according to the light regime.
Photoperiod may play a significant role in the initiation of neuroendocrine
cascades (on–off mechanism) in Antarctic krill, as it has been found to be the
primary signal initiating diapause, migration, or reproduction in other
arthropods (Bradshaw and Holzapfel 2007). It remains to be clarified if the
photoperiodic time measurement inducing seasonal events in Antarctic krill is
related to the circadian clock (Hut et al. 2013; Meuti et al. 2015) or
represents an independent circannual timing system. Using light regime as a
seasonal zeitgeber makes ecologically sense because it is a more reliable cue
than food availability. The intensity of the initiated seasonal physiological
processes may be regulated in the field by the interaction with other factors
such as food or temperature. High food quality and quantity were found to
advance growth (Ross et al. 2000; Atkinson et al. 2006) and maturation (Quetin
and Ross 2001) in Antarctic krill. We pro- pose that this effect is restricted
to specific seasonal periods that are determined by the response of Antarctic
krill’s endogenous timing system to the exposed latitudinal light regime. 

This study has high relevance for future modelling approaches of Antarctic
krill densities in the Southern Ocean, especially under the aspect of climate
change. Recent Antarctic krill models have focussed on intraspecific food
competition (Ryabov et al. 2017) or have been conducted on a conceptual basis
(Groeneveld et al. 2015). The incorporation of light regime into dynamic models
may significantly improve the predictability of growth, energy budget, and
reproduction in Antarctic krill. Recently, a coupled energetics and moult-cycle
model has been developed for Antarctic krill that considered resource
allocation based on the seasonal cycles of growth and maturity (Constable and
Kawaguchi 2018). Further research on the phenology and biological clock of
Antarctic krill will help to better understand its adaptive potential to
environmental changes. 

\section{Conclusion}

This study aimed to investigate the impact of light regime on Antarctic krill’s
phenology and the role of its endogenous timing system. Our observations
suggest that light regime affects seasonal cycles of growth, feeding, lipid
metabolism, and maturity under constantly high food supply. Antarctic krill
possesses an endogenous timing system that maintains seasonal rhythms un- der
constant darkness and is most likely entrained by light regime. Varying
critical photoperiods under different latitudinal light regimes indicate that
this timing system is flexible, allowing Antarctic krill to adjust its
physiological and behavioural responses to the extreme light conditions in the
Southern Ocean. 

\section{Acknowledgements}

We thank the staff at the Australian Antarctic Division (namely R. King, T.
Waller, A. Cooper, and B. Smith) for their advice and the maintenance of the
Antarctic krill during the long-term experiments in the Antarctic krill
aquarium. Sincere thanks go to F. Piccolin and F. Müller for their help in
setting up the experiment and their collegial support. M. Vortkamp is
acknowledged for her help in the laboratory during lipid content analysis. This
study was funded by the Helmholtz Virtual Institute “PolarTime” (VH-VI-500:
Biological timing in a changing marine environment — clocks and rhythms in
polar pelagic organisms), the ministry of science and culture (MWK) of Lower
Saxony, Germany (Research Training Group “Interdisciplinary approach to
functional bio- diversity research” (IBR)), and Australian Antarctic Program
Project \#4037. Additional funds were made available via the PACES (Polar
Regions and Coasts in a changing Earth System) programme (Topic 1, WP 5) of the
Helmholtz Association. 


\section{References}
> see original publication
