% Chapter Template

\chapter{General Discussion} % Main chapter title

\label{Discussion} % Change X to a consecutive number; for referencing this chapter elsewhere, use \ref{ChapterX}

This dissertation gives new insights in the flexible seasonal timing system of
Antarctic krill at different latitudes in the Southern Ocean and in the
molecular mechanisms underlying the seasonal cycles of Antarctic krill. First,
I will discuss how different environmental factors influence seasonal processes
in Antarctic krill with special focus on the effect of latitudinal light
regime. Then, I will move on to the endogenous timing of Antarctic krill
referring to potential mechanisms of photoperiodic entrainment, the molecular
basis of seasonal time measurement and photoperiodic plasticity. Subsequently,
I will give insights in regulatory pathways that may control seasonal processes
in Antarctic krill focussing on genes related to the visual transduction
system, the circadian clock, insulin and juvenile hormone-like signalling, and
the metabolism of signalling compounds in Antarctic krill.

%----------------------------------------------------------------------------------------
%	SECTION 1
%----------------------------------------------------------------------------------------

\section{The effect of latitudinal light regime on the seasonal cycle of \textit{E. superba}}

From field studies, we know that Antarctic krill is able to synchronize its
seasonal physiological functions to the extreme seasonal changes of
photoperiod, food regime and sea ice conditions in the Southern Ocean by having
evolved different overwintering strategies \citep{ericson_seasonal_2018,
hellessey_seasonal_2018, meyer_seasonal_2010, seear_seasonal_2012,
siegel_krill_2012}. However given the large circumpolar distributional range of
Antarctic krill, it is still a matter of discussion which environmental factors
shape its seasonal patterns of physiology and behaviour in the different
latitudinal habitats of Antarctic krill in the field
\citep{kawaguchi_male_2007, schmidt_feeding_2014, seear_seasonal_2012,
spiridonov_spatial_1995}. Variable environmental conditions in the
distributional range of Antarctic krill include local differences of ambient
water temperature, food availability and latitudinal light regime. The
potential involvement of these environmental factors in the regional adaptation
and the regulation of seasonal cycles in Antarctic krill will be discussed in
detail in the following paragraphs.

Ambient water temperature may influence local differences in Antarctic krill
growth, maturity and metabolic activity, but is most likely not the main
environmental cue triggering Antarctic krill's seasonal cycle of growth,
maturity and metabolism. Higher water temperatures may increase the metabolic
activity of Antarctic krill \citep{segawa_oxygen_1979} and its growth rates by
shortening the intermoult period \citep{brown_temperature_2010,
bucht_isolation_1991}.  Different temperature regimes also led to differences
in the timing of the shortening of the intermoult period during the maturity
cycle of Antarctic krill \citep{kawaguchi_male_2007}. Seasonal differences in
the intermoult period of Antarctic krill were often linked to the seasonal
cycle of water temperature (e.g. \citet{kawaguchi_modelling_2006,
tarling_satiation_2006}. However, long-term laboratory experiments under
constant food supply and temperature rather suggest that the general seasonal
patterns of growth, maturity and metabolic activity are mediated by light
regime and an endogenous timing system in Antarctic krill
\citep{brown_temperature_2010, brown_flexible_2011, brown_long-term_2013}.
Results from this dissertation (Publication 1) further suggest that temperature
is not the main factor shaping the seasonal differences in gene expression
between summer and winter Antarctic krill in the different latitudinal habitats
of Antarctic krill, because least differences were observed in the low-latitude
region South Georgia where seasonal temperature differences are the strongest
\citep{whitehouse_seasonal_1996}.

Seasonal and regional differences in food quantity and quality may affect the
seasonal physiological and behavioural processes in Antarctic krill. The
variable timing of phytoplankton blooms in the different latitudinal habitats
of Antarctic krill has been discussed in relation to regional differences in
the reproductive timing of Antarctic krill \citep{spiridonov_spatial_1995} and
the timing of its main growth period \citep{kawaguchi_modelling_2006}. Indeed,
it has been shown in laboratory studies that high food supply
\citep{buchholz_moult_1991} can accelerate growth and maturation in Antarctic
krill \citep{kawaguchi_male_2007}. In the field, differences in food
availability have also been associated with the observed regional variation of
feeding behaviour in Antarctic krill in winter \citep{schmidt_feeding_2014}. We
explained the comparatively low  seasonal differences in gene expression in the
low-latitude region South Georgia by the milder seasonal conditions including
effects of the low-latitude light regime (16 h light in mid-summer and even 9 h
light in mid-winter, see Figure 1b), no sea ice cover in winter and
consequently the generally higher food availability in that region (Publication
1).

Indeed, results from our controlled laboratory study suggest that feeding and
digestion in Antarctic krill might be controlled by a combination of light
regime and food supply (Publication 2 \& 3). We found that light regime
generally affected the seasonal pattern of feeding index and the expression of
feeding-related genes in Antarctic krill indicating a slightly reduced feeding
activity during winter. However, the field observations of extremely low
feeding activity of Antarctic krill during winter \citep{meyer_seasonal_2010}
and local differences of winter feeding behaviour \citep{schmidt_feeding_2014}
were not confirmed under the simulation of different latitudinal light regimes
and constant food supply in the laboratory. This indicates that the regional
feeding behaviour of Antarctic krill is probably largely dependent on regional
food availability.

This dissertation shows that light regime, and in particular changes in
latitudinal light regime, play a major role for the synchronization of
Antarctic krill's seasonal cycles to the variable seasonal conditions in its
different latitudinal habitats of the Southern Ocean. For the first time, a
controlled two-year laboratory experiment was conducted that simulated the
latitudinal light regimes 52$^{\circ}$S and 66$^{\circ}$S, and constant
darkness under stable food supply and temperature conditions (Publication 2 \&
3). This two-year laboratory experiment partly confirms results from other
shorter laboratory studies showing that light regime is the major cue that
entrains seasonal cycles of maturity \citep{brown_flexible_2011}, growth,
metabolic activity and gene expression in Antarctic krill
\citep{piccolin_seasonal_2018}. Under controlled laboratory conditions, light
regime stimulates the different overwintering mechanisms (an their recovery
towards summer) of Antarctic krill that are also observed in the field:
including the regression of the outer sexual organs \citep{siegel_krill_2012},
the stagnation of growth or shrinkage \citep{meyer_seasonal_2010,
quetin_behavioral_1991}, metabolic depression and reduced gene expression
during winter \citep{meyer_seasonal_2010, seear_seasonal_2012}, and the
seasonal accumulation of lipid stores \citep{meyer_seasonal_2010}. In
particular, publication 2 revealed a seasonal cycle of sexual maturity and
growth under both simulated latitudinal light regimes over a two-year period,
and gave novel insights in the effect of latitudinal light regime on the
seasonal cycle of feeding, lipid content and the critical photoperiod for
maturity in Antarctic krill. Publication 3 showed that different latitudinal
light regimes could flexibly affect the seasonal expression patterns of
different metabolic and regulatory genes in Antarctic krill. 

The following observations support our hypothesis that latitudinal light regime
affects the seasonal cycle of Antarctic krill (Publications 2 \& 3). 

Results from publication 2 \& 3 suggest that the seasonal cycle of lipid
metabolism is regulated and adjusted according to the latitudinal light regime.
Under the simulated high-latitude light regime 66$^{\circ}$S, a pronounced
seasonal cycle of lipid content as well as an enhanced autumn expression of
genes possibly related to lipid accumulation was observed. This may be a
potential adaptation mechanism of Antarctic krill that is regulated by the
high-latitude light regime leading to larger lipid stores that are required to
survive in regions at higher latitudes, e.g. Lazarev Sea, that are
characterized by near-constant darkness and extremely low food availability
during winter.  Therefore, differences in latitudinal light regime may partly
explain why winter krill from the low-latitude region South Georgia was found
to have lower lipid stores than Antarctic krill from the higher latitudinal
regions Bransfield Strait and Lazarev Sea \citep{schmidt_feeding_2014}.

On gene expression level, the effect of latitudinal light regime on Antarctic
krill further manifested. We observed the strongest seasonal changes in gene
expression under the simulated high latitudinal light regime 66$^{\circ}$S,
whereas these differences seemed to be weaker under the low-latitude light
regime 52$^{\circ}$S (Publication 3). Similar observations were made in
Antarctic krill from the field where the least seasonal differences in gene
expression were observed in the low-latitude region South Georgia
(54$^{\circ}$S) with respect to the higher latitudinal regions South
Orkneys/Bransfield Strait and Lazarev Sea (60$^{\circ}$S-66$^{\circ}$S)
(Publication 1). Therefore, we suggest that latitudinal light regime is an
important cue for Antarctic krill to be able to regionally adjust its seasonal
patterns of gene expression in the field.

To which extent the seasonal cycles of metabolism, growth, maturity, lipid
utilisation and feeding in Antarctic krill are directly affected by the
latitudinal light regime or indirectly modulated by the interaction with other
internal regulatory processes, it is not yet clear. This comprises the open
question if seasonal changes in metabolic activity may affect feeding activity
in Antarctic krill, or the other way around \citep{teschke_simulated_2007}. The
seasonal maturity cycle of Antarctic krill also seems to partly interact with
seasonal growth patterns \citep{tarling_growth_2016, thomas_thelycum_1987} and
the seasonal utilisation of lipid stores \citep{teschke_effects_2008}.
Moreover, it is known that seasonal processes in Antarctic krill are regulated
by an endogenous timing system that will be discussed in the following section.

%----------------------------------------------------------------------------------------
%	SECTION 2
%----------------------------------------------------------------------------------------

\section{Endogenous timing of seasonal processes in \textit{E. superba}}

This dissertation complements findings from earlier studies revealing that
seasonal cycles of maturity, growth and metabolic activity are regulated by an
endogenous timing system in Antarctic that is likely synchronized by light
regime \citep{brown_flexible_2011, brown_long-term_2013,
piccolin_seasonal_2018}. During our controlled two-year laboratory experiment,
we show that seasonal patterns of growth, feeding and maturity persist in
Antarctic krill exposed to constant darkness (Publication 2) which is an
indication for the presence of a biological (or internal) clock controlling
these seasonal processes \citep{visser_phenology_2010}.  Circannual clocks
control the internal timing of seasonal life cycle events and they are
typically entrained by photoperiodic cues \citep{helm_annual_2013}. This is
also likely the case for Antarctic krill, because light regime has been found
to be the major cue for the stimulation of circannual rhythms in Antarctic
krill (discussed in detail above). Antarctic krill's susceptibility to other
environmental factors such as temperature, food supply or social cues may be
restricted to particular periods of the underlying clock-controlled rhythms in
Antarctic krill, a mechanism also known in other animals
\citep{visser_phenology_2010}.  Interestingly, social cues have been found to
influence the seasonal behaviour and reproduction of birds
\citep{helm_sociable_2006}. A similar mechanism of social interaction may also
play a role for the adjustment of seasonal processes in the different
latitudinal habitats of Antarctic krill, such as the synchronization of the
spawning season.

Considering the huge latitudinal range of Antarctic krill and the corresponding
differences in seasonal day length, Antarctic krill must have evolved a highly
flexible clock machinery to ensure survival under these extreme light
conditions.

In publication 2, we used the concept of critical photoperiod to emphasize
Antarctic krill's flexibility in photoperiodic timing under different
latitudinal light regimes. Interestingly, the critical photoperiod for maturity
in Antarctic krill was found to be higher under the high-latitude light regime
66$^{\circ}$S with respect to the low-latitude regime 52$^{\circ}$. The
increasing trend in critical photoperiod with latitude in Antarctic krill seems
to be an adaptation to the more extreme light conditions at higher latitudes
where Antarctic krill need to enter the state of sexual regression and
re-maturation at comparatively longer photoperiods than at lower latitudes.
Similar adaptations have been observed in insects for diapause initiation that
showed a similar increasing trend of critical photoperiod with latitude
\citep{brandt_biodiversity_2007, hut_latitudinal_2013,
tyukmaeva_adaptation_2011}

Which specific light cues are required to synchronize Antarctic krill's
seasonal timing system in different latitudinal regions is not yet clear. A
concept of photoperiodic entrainment has been suggested for the seasonal cycles
of maturity and metabolic activity in Antarctic krill
\citep{brown_flexible_2011, piccolin_seasonal_2018}. From observations under
photoperiodic controlled laboratory experiments, these authors suggest that the
timing of sexual regression and the reduction of metabolic activity towards
winter are dependent on a decrease in day length, whereas sexual re-maturation
and metabolic recovery towards summer are initiated by the endogenous timing
system of Antarctic krill independent of light cues. A similar mechanism could
have triggered the endogenous cycle of maturity, and the more robust endogenous
rhythms of growth and feeding in Antarctic krill under constant darkness of our
two-year laboratory experiment (Publication 2). However, the study by
\citet{hirano_antarctic_2003} also indicates that maturation and spawning in
Antarctic krill may be triggered by a dark period followed by a period of long
photoperiods.  Similar photoperiodic cues are required in some bird species for
the initiation of the breeding season \citep{helm_annual_2013}. These findings
suggest that the seasonal re-maturation of Antarctic krill may be controlled by
a combination of endogenous cues that may mostly trigger the re-maturation of
the outer sexual traits and additional light cues that may induce
vitellogenesis and spawning in Antarctic krill.

The molecular mechanisms that control the seasonal timing of physiological and
behavioural responses are still poorly understood in animals
\citep{helm_annual_2013}. Conceptionally, the photoperiodic signal is perceived
by an internal photoperiodic timer that regulates gene expression and
neuroendocrine signalling, thereby initiating complex physiological,
reproductive and developmental processes \citep{brandt_biodiversity_2007,
visser_phenology_2010}.

A recent study suggests that the circadian clock may be involved in the
regulation of the seasonal cycles of Antarctic krill
\citep{piccolin_seasonal_2018}.  Under photoperiodic-controlled laboratory
experiments, \citet{piccolin_seasonal_2018} found a seasonal  expression
pattern of the clock genes \textit{clk}, \textit{cry2}, and \textit{tim1} which
was mostly similar to our observations made in publication 3. There is evidence
from RNAi experiments that the circadian clock as a functional entity may
control the photoperiodic initiation of reproductive diapause in insects
(reviewed by \citet{meuti_evolutionary_2013, meuti_functional_2015}). A similar
mechanism may be present in Antarctic krill, given that the circadian
clock-work in Antarctic krill is functionally characterized and comprises a
photoperiod-induced entraining mechanism \citep{biscontin_functional_2017}.
However, the functional involvement of the circadian clock in the timing of
seasonal cycles of Antarctic krill still has to be proven in knock-out
experiments.

On the other hand, Antarctic krill may have evolved a seasonal timing mechanism
that is independent of the circadian timing system, such as in the
pitcher-plant mosquito \citep{emerson_evolution_2009}.
\citet{emerson_evolution_2009} also states that this does not exclude the
possibility that individual clock genes are involved in the seasonal time
measurement. Other potential seasonal timing mechanisms include the regulation
by non-coding RNAs and epigenetic modification (Helm and Stevenson, 2014). Both
processes have been studied in hibernating mammals and were associated with the
regulation of various processes such as lipid metabolism and the timing of
reproductive development \citep{lang-ouellette_mammalian_2014,
stevenson_epigenetic_2017, stevenson_reversible_2013}.

Which molecular factors promote the seasonal and regional photoperiodic
plasticity in Antarctic krill are unknown. Clock gene polymorphisms have been
linked to the photoperiodic plasticity of seasonal timing in insects, birds and
fish \citep{caprioli_clock_2012, hut_latitudinal_2013, omalley_clock_2010}.
Genetic variation, such as clock gene polymorphisms, may also play a role for
latitudinal adaptation of the seasonal response of Antarctic krill in the
field. However in the same population of Antarctic krill, different seasonal
gene expression patterns were detected during our simulation of the latitudinal
light regimes 66$^{\circ}$S and 52$^{\circ}$S (Publication 3). These findings
likely suggest a general plasticity of the seasonal timing system of Antarctic
krill independent of genetic variation. The visual plasticity within the
extreme light conditions of the Southern Ocean may be linked to the flexible
'light' adjustment of components within the visual perception system of
Antarctic krill similar to the adaptive mechanisms explained for the circadian
visual system in arthropods by \citet{mazzotta_cry_2010}. The discovery of the
versatile opsin photopigments of Antarctic krill \citep{biscontin_opsin_2016}
and the recent observation of seasonal expression patterns of circadian-related
opsins under a simulated light regime \citep{piccolin_seasonal_2018} further
support the assumption that the visual perception system may play a major role
in the regulation of the flexible phenology of Antarctic krill.

\section{Insights into the regulatory pathways mediating the seasonality in \textit{E. superba}}

To identify regulatory processes that are potentially involved in the seasonal
timing and latitudinal adaptation of Antarctic krill, we used an RNAseq
approach analysing regional and seasonal differences in gene expression of
summer and winter krill from three different latitudinal regions: South Georgia
(54$^{\circ}$S), South Orkneys/Bransfield Strait (60$^{\circ}$S-63$^{\circ}$S)
and Lazarev Sea (62$^{\circ}$S -66$^{\circ}$S) (Publication 1). Thereby, we
were not only able to characterize seasonal gene expression in Antarctic krill
under variable environmental conditions, but also to propose various target
genes that are possibly involved in the regulation of seasonal processes in
Antarctic krill. These target genes may be a starting point to elucidate the
molecular mechanisms of the seasonal timing system in Antarctic krill. In
publication 3, we chose a few of these target genes, in combination with other
genes of interest, to clarify the effect of latitudinal light regime on the
expression of these genes. We showed that light regime affected genes with
functions in glycolysis and the citric acid cycle, the respiratory chain, lipid
metabolism and transport, prostaglandin biosynthesis, the activation of
neuropeptides, translation, the digestion of peptides and the circadian clock
(Table \ref{discussion_table_1}).

%%%% TODO: Table 1
% Please add the following required packages to your document preamble:
% \usepackage{booktabs}
% \usepackage{graphicx}
\begin{table}[]
\centering
\caption{List of seasonally expressed genes with functional process that were
found to be significantly rhythmic over a period of 12 months under the light
regime 66$^{\circ}$S using RAIN (Publication 3).}
\label{discussion_table_1}
{\scriptsize
\begin{tabular}{@{}ll@{}}
\toprule
\textbf{Gene}                                     & \textbf{Functional process}       \\ \midrule
\textit{phosphofructokinase-6}                    & glycolysis                        \\
\textit{glyceraldehyde-3-phosphate dehydrogenase} & glycolysis                        \\
\textit{citrate synthase}                         & citric acid cycle                 \\
\textit{ATP synthase subunit gamma}               & respiratory chain                 \\
\textit{cytochrome c oxidase subunit NDUFA4}      & respiratory chain                 \\
\textit{carbonyl reductase 1}                     & prostaglandin biosynthesis        \\
\textit{hematopoietic prostaglandin D synthase}   & prostaglandin biosynthesis        \\
\textit{neuroendocrine convertase 1}              & activation of neuropeptides       \\
\textit{fatty acid binding protein}               & lipid transport                   \\
\textit{fatty acid synthase}                      & fatty acid synthesis              \\
\textit{galactosylceramide sulfotransferase}      & sulfation of membrane glycolipids \\
\textit{ribosomal protein S13}                    & protein synthesis                 \\
\textit{ribosomal protein S18}                    & protein synthesis                 \\
\textit{anionic trypsin 1}                        & digestion of peptides             \\
\textit{timeless1}                                & circadian clock                   \\ \bottomrule
\end{tabular}%
}
\end{table}

RNAseq target gene selection was carried out from differentially expressed
genes between summer and winter krill including the following gene categories:
metabolism related to bioactive lipids, hormone metabolism, visual perception,
receptor-related proteins, development, reproduction, dephosphorylation and
transcriptional regulation (Publication 1).

In the following, I will focus on the most relevant target genes and regulatory
processes that are connected to a) visual perception, b) the circadian clock,
c) the insulin signalling and the juvenile-hormone like pathway, and d) the
metabolism of other signalling compounds and hormones. If appropriate, they
will be discussed in relation to findings from our controlled laboratory
experiments (Publication 3).

With respect to the visual perception system of Antarctic krill, we identified
genes coding for the signal transduction-related protein arrestin and the
enzyme carotenoid isomerooxygenase (gene = \textit{ninaB}) (Publication 1). The
carotenoid isomerooxygenase catalyses the biogenesis of photopigments (visual
opsins) \citep{voolstra_ninab_2010}. In \textit{Drosophila}, The absorption of
light by visual pigments causes a conformational change that activates
G-proteins and thereby initiates the phototransduction cascade
\citep{mazzotta_circadian_2016}.  Arrestins are involved in the termination of
the phototransduction cascade by inhibiting the interaction between visual
pigments and G-proteins \citep{montell_drosophila_2012}. A daily bimodal gene
expression pattern of arrestin has already been identified in Antarctic krill
in the field \citep{de_pitta_antarctic_2013}. Therefore, these genes may be
interesting candidates for the investigation of seasonal and regional
differences in visual transduction system in Antarctic krill.

Seasonal photoperiodic cues may be transmitted to the circadian clock system
that may also play a role for the timing of seasonal processes in Antarctic
krill (as discussed above). In the field, we found pronounced seasonal
differences in the expression of genes that have known connections to the
circadian clock and therefore we suggest that these genes may additionally have
seasonal regulatory roles in Antarctic krill (Publication 1). 


We identified a target gene coding for the endopeptidase neprilysin-1
(Publication 1). Neprilysin-like peptidases are involved in in the breakdown of
neuropeptides, and have been found to terminate synaptic signalling of the
circadian neurotransmitter pigment dispersing factor (PDF)
\citep{isaac_metabolic_2007}. In insects, it has been  proposed that
PDF-neuroactive neurons play a role for the transmission of photoperiodic
signals that initiate the diapause in insects \citep{hamanaka_synaptic_2005,
ikeno_involvement_2014}. Moreover, neprilysins have important functions in the
regulation of the reproductive processes of \textit{Drosophila}
\citep{sitnik_neprilysins:_2014}.

Various genes that coded for receptor-related proteins were found (Publicaton
1). These candidate genes included the leucine-rich repeat-containing G-protein
coupled receptor 4 (protein LGR4) that has been associated with the circadian
regulation of plasma lipids in mice \citep{wang_lgr4_2014}. In addition, LGR4
plays a role in the regulation of Wnt/$\beta$-catenin signalling
\citep{carmon_r-spondins_2011} that has known functions in various
developmental processes in arthropods \citep{murat_function_2010}. Therefore,
LGR4 may be an interesting target to study its possible role in the seasonal
developmental processes and lipid metabolism of Antarctic krill.

Another potential target gene may be the serine/threonine-protein phosphatase
2A (PP2A) (Publication 1) that is an important post-translational regulator of
the circadian clock system \citep{pegoraro_animal_2011}. Interestingly, PP2A is
also involved in the regulation of visual transduction in Drosophila
\citep{wang_role_2008} and the ovarian maturation in crustaceans
\citep{zhao_molecular_2017}.

We also identified a gene coding for the CREB-binding protein (Publication 1)
which is a transcriptional co-regulator of the circadian clock and
independently affects circadian locomotor activity in \textit{Drosophila}
\citep{maurer_creb-binding_2016}. The CREB-binding protein is also involved in
the transcriptional regulation of various developmental processes in
\textit{Drosophila}, and in particular the juvenile hormone signalling pathway
\citep{roy_multiple_2017}.

In insects, insulin and juvenile hormone signalling have been found to
stimulate reproductive development, growth and metabolism, whereas the
interruption of these pathways leads to diapause formation that is
characterized by lipid accumulation, reproductive arrest and supressed
metabolism \citep{flatt_hormonal_2005, hahn_energetics_2011, liu_absence_2017,
schiesari_insulin-like_2016, sim_insulin_2013}. We suggest that similar
regulatory pathways may be important for the regulation of seasonal cycles of
reproduction, growth and metabolism in Antarctic krill, especially with respect
to its pronounced overwintering mechanisms. A gene coding for an insulin-like
peptide has been identified in Antarctic krill in the field by
\citet{seear_seasonal_2012} who discussed its function with respect to the
reproductive physiology of Antarctic krill in summer. We add several target
genes that are related to insulin signalling and have juvenile-hormone like
functions in Antarctic krill (Publication 1).

Regarding insulin signalling, we identified a gene coding for the adiponectin
receptor (Publication 1). In \textit{Drosophila}, the adiponectin receptor is
involved in the regulation of insulin secretion and respectively controls
glucose and lipid metabolism \citep{kwak_drosophila_2013}, whereas in crustaceans it has
been found to play a role in the maintenance of skeletal muscles \citep{kim_molecular_2016}.

The transcriptional regulator Krüppel homolog 1 may be another target gene with
potential similarities to the juvenile hormone signalling pathway (Publicaton
1). In insects, Krüppel homolog 1 acts as a mediator of the  juvenile hormone
signal, thereby affecting vitellogenesis and oocyte maturation
\citep{song_kruppel-homolog_2014} and development \citep{minakuchi_kruppel_2008}.

Moreover, we propose that genes related to the metabolism of methyl farnesoate
may be involved in the seasonal regulation of moulting and reproduction in
Antarctic krill (Publication 1 \& 3). Methyl farnesoate is a precursor of the
insect juvenile hormone III and it is considered to be the 'crustacean juvenile
hormone' \citep{homola_methyl_1997}. Methyl farnesoate has known functions in
the stimulation of moulting and reproduction in crustaceans
\citep{reddy_involvement_2004}.  In publication 1, we identified target genes
coding for juvenile hormone esterase-like carboxylesterases which are
potentially involved in the inactivation of methyl farnesoate
\citep{lee_two_2011}. Under our three simulated light regimes (Publication 3),
we also investigated the gene expression profile of the farnesoic acid
O-methyltransferase (gene famet) which catalyses the biosynthesis of methyl
farnesoate in crustaceans \citep{gunawardene_function_2002}. We found an
upregulation of famet under constant darkness during most of the study period
(except autumn), probably due to generally higher levels of methyl farnesoate
secretion in the absence of light cues (Publication 3). We suggest that
constant darkness disrupts the photoperiodic-controlled inhibitory mechanism
that controls methyl farnesoate secretion in crustaceans
\citep{swetha_reproductive_2011, tsukimura_regulation_1992}. The consequently
elevated levels of methyl farnesoate secretion may be partly responsible for
the dampening of the seasonal maturity cycle of Antarctic krill under constant
darkness \citep{horing_light_2018}.

Last but not least, we identified several other target genes that code for
enzymes involved in the metabolism of hormones and bioactive compounds, such as
steroid, octopamine and thyroxine metabolism, and may have regulatory roles in
seasonal processes in Antarctic krill (Publication 1). Ecdysteroids and
vertebrate-type steroids have known functions in the regulation of moulting and
reproduction in crustaceans \citep{lafont_steroids_2007}. The aminergic
neurotransmitter octopamine has been found to stimulate heart beat and
behaviour in lobsters \citep{battelle_targets_1978, kravitz_hormonal_1988},
whereas it inhibits dormancy in \textit{Drosophila}
\citep{andreatta_aminergic_2018} and may have similar effect on Antarctic krill
in its 'active' seasons. In vertebrates, thyroid hormone (thyroxine as
precursor) and its metabolism are important for the seasonal timing of
reproduction \citep{nishiwaki-ohkawa_molecular_2016,
saenzdemiera_circannual_2014}. In insects however, the functional relevance and
mechanism of thyroid hormone signalling is not clear
\citep{flatt_comparing_2006}.

Interestingly, our selected target genes also included the \textit{prohormone
convertase 1 (nec1)} coding for a prohormone processing enzyme, and
\textit{hematopoietic prostaglandin D synthase (hpgds)} and \textit{carbonyl
reductase 1 (cbr1)} that are involved in the biosynthesis of prostaglandins
(Publication 1). For the first time, these genes were investigated during our
long-term laboratory study and showed significant seasonal rhythmicity under
the simulated light regime 66$^{\circ}$S. These results suggest the photoperiodic
control of their seasonal expression patterns and a de-synchronization of these
seasonal patterns under constant darkness (Publication 3).

Prohormone convertases that catalyse the activation of hormones and
neuropeptides \citep{zhou_proteolytic_1999} may play an important role in the seasonal
regulation of growth, reproduction and metabolism in Antarctic krill.
Prohormone convertases have been found in the neuroendocrine tissue of the
crustacean eyestalk including the X-organ-sinus gland complex \citep{tangprasittipap_structure_2012, toullec_molecular_2002} that is the major production site for many
neuropeptides such as the crustacean hyperglycemic hormone (CHH)
\citep{nagaraju_reproductive_2011}. From our laboratory observations
(Publication2), we may suggest that a photoperiodic-controlled signalling
pathway in the eyestalk regulates the seasonal expression profile of
\textit{nec1} in Antarctic krill which in turn may affect the activation of
various hormones that have known regulatory roles in reproduction, growth and
metabolism \citep{nagaraju_reproductive_2011}.

Prostaglandins are physiologically active lipid compounds that are found in
various crustacean tissues including the X-organ-sinus gland, the brain and the
ovary \citep{nagaraju_reproductive_2011}. In crustaceans, prostaglandins play a
major role in ovarian development \citep{wimuttisuk_insights_2013} and may have
similar effects on the reproductive physiology of Antarctic krill. Our
laboratory results (Publication 3) give new insights into the the
transcriptional regulation of the biosynthesis pathway of prostaglandins that
is mediated by seasonal light cues in Antarctic krill.

\section{Conclusions and outlook}

This dissertation gave new insights in the effect of different latitudinal
light regimes on the seasonal cycle of Antarctic krill and the potential
internal regulatory mechanisms of its seasonal timing system. In publication 1,
we explored the seasonally differential gene expression patterns in Antarctic
krill in different latitudinal regions in the Southern Ocean and found that
krill from the low-latitude region South Georgia showed the least seasonal
differences, probably due to Antarctic krill's flexible adaptation to the
low-latitude light regime and enhanced food availability in this region.
Moreover, we detected several target genes that may be the basis for future
studies of the internal mechanisms governing the seasonal cycles in Antarctic
krill. To further elucidate the role of latitudinal light regime and of the
endogenous timing system on the phenology in Antarctic krill, we conducted the
first two-year laboratory experiment under constant food and temperature
conditions simulating the latitudinal light regimes 52$^{\circ}$S,
66$^{\circ}$S, and constant darkness (Publication 2 \& 3).

Publication 2 highlighted that the simulated latitudinal light regimes affected
the seasonal cycles of growth, feeding, lipid metabolism and maturity in
Antarctic krill with evidence that an endogenous timing system was controlling
these seasonal processes (except for seasonal cycle of lipid content). We
emphasized the photoperiodic flexibility of Antarctic krill by showing that the
critical photoperiod for maturity is higher under the high-latitude light
regime 66$^\circ$S with respect to the low-latitude light regime 52$^{\circ}$S.
Antarctic krill's flexible response to different latitudinal light regimes was
further manifested on gene expression level that was investigated in
publication 3.  These results also suggested the involvement of genes related
to the circadian clock, the activation of neuropeptides, and the metabolism of
methyl farnesoate and prostaglandins in the light-dependent regulation of
seasonal processes in Antarctic krill.

Our findings point to a highly flexible seasonal timing system that enables
Antarctic krill to adjust its seasonal photoperiodic response to extreme
differences in latitudinal light regime. In the field, the underlying
'clock-controlled' seasonal cycles of physiology seem to be adjusted by other
environmental factors such as temperature and food supply which further
promotes the regional acclimatization of Antarctic krill.

It is not yet clear if Antarctic krill is able to adjust its phenology to
anthropogenic environmental changes in the Southern Ocean. Its flexible timing
system may be an advantage under the observed southward shift of Antarctic
krill in the Southwest Atlantic Sector \citep{atkinson_krill_2019}, because
Antarctic krill is already well adapted to the extreme light regimes at higher
latitudes. However, it is still unknown if Antarctic krill will be able to
adapt in time to climate change-related alterations in sea-ice dynamics and the
timing of phytoplankton blooms. Further research is required to better
understand Antarctic krill's flexibility in responding to variable food
conditions in different seasons in conjunction with its photoperiodic
controlled seasonal timing system. The findings of this dissertation are highly
relevant for future modelling approaches that may help to predict potential
scenarios how Antarctic krill will respond to climate change. The prediction of
growth, energy budget and reproduction may be significantly improved by
incorporating the factor latitudinal light regime as the main driver for
underlying seasonal cycles of growth, feeding, lipid metabolism and maturity in
Antarctic krill. In addition, the effect of regionally variable factors such as
temperature and food availability may be added during particular periods of the
seasonal cycle.

We are just in the beginning of understanding the molecular mechanisms that
control seasonal physiological cycles in Antarctic krill. Antarctic krill may
be an interesting target organism to understand the flexibility of seasonal
timing systems. Future laboratory experiments may test the combined effects of
different environmental factors on seasonal processes in Antarctic krill, such
as light regime, temperature and food supply. The selected target genes from
our seasonal RNAseq study (Publication 1) may be a starting point for further
molecular investigations such as testing the influence of light regime on their
seasonal expression patterns. An important step towards the functional
understanding of seasonal timing in Antarctic krill may be the establishment of
gene silencing techniques such as RNAi that could for instance solve the
question if circadian clock genes are involved in the seasonal regulatory
processes in Antarctic krill. Further studies should also focus on the effects
of endocrine signalling pathways in Antarctic krill which are not well
understood and are an important link for the regulation of seasonal processes
in Antarctic krill.

\printbibliography[heading=subbibliography]
